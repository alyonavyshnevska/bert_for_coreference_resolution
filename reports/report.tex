\documentclass[11pt]{article}

\usepackage[scaled]{helvet}
\renewcommand\familydefault{\sfdefault} 
\usepackage[T1]{fontenc}%%% use with PDFLaTeX
\usepackage[utf8]{inputenc}
\usepackage[top=25mm,left=25mm,right=25mm,bottom=20mm,headsep=15pt,footskip=15pt,a4paper]{geometry} % see geometry.pdf on how to lay out the page. There's lots.
\usepackage[english]{babel}

\usepackage{color}
\usepackage{enumitem}
\usepackage[colorlinks,citecolor=blue]{hyperref}
\usepackage[parfill]{parskip} % no indentation

%\usepackage{helvet}
\usepackage{color}

\usepackage[backend=biber,style=authoryear]{biblatex}
%\usepackage{biblatex}
\addbibresource{references.bib}


\makeatletter         
\def\@maketitle{   % custom maketitle 
\begin{center}%
{\bfseries \@title}%
{\bfseries \@author}%
\end{center}%
\smallskip \hrule \bigskip }


\title{{\LARGE Coreference Resolution}\\[1.5mm]
{\large Working Title, First Draft: Content, not Wording}\\[1.5mm]} %%% Span Representation And Long-Range Coreference
\author{Patrick Kahardipraja, Olena Vyshnevska}%%% Fill in your name 
\date{} % delete this line to display the current date


%%% BEGIN DOCUMENT
\begin{document}

\maketitle


\section{Introduction}

\subsection{Role of Coreference Resolution in NLP}
Some nice long paragraphs about place of coreference in NLP  

\subsection{Challenges in the Field}
Paragraphs about problems with coref   
Narrow down to loca/non-local coreference . 

\subsection{Our Proposed Solution}
% Very broadly what we intended to to

Since the introduction of pre-trained BERT model  \parencite{devlin2019bert} there has been a lot of research showing how the model outperforms previous state-of-the-art task-specific NLP models. % cite more papers here 
\textcite{joshi2019coref} has introduced BERT models into coreference resolution tasks. However, the researchers have also encoded other information on top of BERT embeddings. 

Our first research question is based on the recent research which shows that the BERT model itself encodes a significant amount of semantic and syntactic information. Therefore, we decided to test how well the span representations using BERT embeddings from the model by \textcite{joshi2019coref} finetuned on Ontonotes corpus can encode coreference resolution information without any layers of additional information about a given sequence. 

As a baseline, we used pre-trained BERT embeddings not finetuned on Ontonotes. To encode a mention span,  we used a convolutional layer and a  self-attention layer.

Our second research question, which naturally follows from the first one, whether the span representations encoded with a pre-trained BERT model simply modelling local coreference? Will it be able to encode non-local coreference dependencies? 

We found that the span represenations from within the coreference model by \textcite{joshi2019coref} reach \textit{insert some numbers}. The baseline model is at \textit{insert numbers}. This findings suggests that the pre-trained BERT model is a powerful tool for encoding information relevant to coreference resolution. However, since the research is heavily based on the English language, it may be a phenomena particular to this language. The nominal declension in Ukrainian, for example, has seven cases; adjectives, pronouns have gender specific forms. Therefore, a similar analysis should be done for other languages before one can generalize the findings universally. 

Our code is publicly available on GitHub.  \footnote{\url{https://github.com/alyonavyshnevska/bert_for_coreference_resolution}}.


\section{Related Work} 

\subsection{Approaches to Coreference Resolution}
some paragraphs about mention mention-ranking, and others

\subsection{State of the Art Coreference Models}
%end-to-end and higher order
%bert, bert for coref

\textcite{devlin2019bert} made a significant break-through with their pre-trained BERT model. It can be finetuned with one additional output layer
to create state-of-the-art models for a wide
range of tasks, such as question answering and language inference, without substantial taskspecific architecture modifications. BERT's training examples consist of 128 and 512 word pieces. Such passage-level training has been an important improvement over the previous methods. It helps the model learn dependencies over text sequences longer than one sentence, such as the previous state of the art model ELMo \parencite{peters2018elmo}.

\textcite{devlin2019bert} have extended the influential ELMo-based coreference model by \textcite{lee2018higher}. The authors have achieved a 3.9\% improvement on OntoNotes by replacing ELMo embeddings with BERT embeddings. 

\subsection{Coreference Arc Prediction Task}
% \textcite{liu2019linguistic}  Ling Knowledge and Transfer, Negative Samples, arc prediction task

 \textcite{liu2019linguistic} found that frozen contextual represenations fed into linear models can show similar levels of performance as state of the art task-specific models on many NLP tasks. The authors use probing models, also known as auxiliary or diagnostic classifiers \parencite{shi2016string, kadar2017representation} to analyse the linguistic information within contextual word representations. To test how the probing model performs in comparison to ELMo, the authors tested the models on a coreference arc prediction task, where the model predicts whether two mentions corefer. The Ontonotes dataset was used. To train the probing model the authors generate negative examples, where mentions that do not corefer are fed into the probing model alongside the mentions that do corefer. In our work we will also build our experiments similar to coreference arc prediction tasks suggested by the authors.

\subsection{Probing Model}

%\textcite{tenney2019context} Probing Model


\textcite{tenney2019context} introduced edge probing task design investigated and found that existing models trained on language modeling and translation produce strong representations for syntactic phenomena, but only offer comparably small improvements on semantic tasks over a non-contextual baseline. For coreference resolution, the models had to determining whether two spans of tokens refer to the same entity. In order to investigate how good the contextual word representation models can model long-range dependencies, the authors used a fixed-width convolutional layer on top of the word representations to build spans of word pieces. The authors concluded that using the CNN layers on top of BERT-large pretarined model has performed particularly well on difficult semantic tasks, such as coreference resolution. Hence, in out work we will also follow the approach of the authors to construct our baseline. 

%\textcite{joshi2019coref} bert for coref

%\textcite{lee2018higher} Higher-Order

%\textcite{lee2017end} End-to-End

%\textcite{devlin2019bert} Bert

\section{Approach}


\subsection{Span Representation and Long-Range Coreference}

We attempt to investigate to what extent the span representations proposed using BERT \parencite{devlin2019bert} embeddings in \textcite{joshi2019coref} can encode coreference information, and whether it is able to encode non-local coreference phenomena or is it just simply modeling local coreference.

In order to analyse this, we consider 2 kinds of span representations:
1. BERT-based span representations finetuned on OntoNotes in \textcite{joshi2019coref} with first and last word-pieces (concatenated with the attention version of all word pieces in the span).
2. pre-trained BERT embeddings (not finetuned on OntoNotes) for all tokens within the mention span, which is then passed through a convolutional layer (with kernel width of 3 and 5) to incorporate the local context and followed by self-attention pooling operator to produce a fixed-length span representations. This is to model head words, inspired by approach from \textcite{tenney2019context}.

\subsection{Arc Prediction Task}

Both span representations will be then used as inputs for coreference arc prediction task \cite{liu2019linguistic}, where a probing model (in this case a simple FFNN) is used to predict coreference relations. The probing model is designed with limited capacity to focus on what information that can be extracted from the span representations. The probing model itself has a sigmoid output layer, which is trained to minimize binary cross entropy. Each negative samples (\textit{w\_entity}, \textit{wb}) will be generated for every positive samples (\textit{wa}, \textit{wb}) where \textit{wb} occurs after \textit{wa} and \textit{w\_entity} is a token that occurs before \textit{wb} and belong to a difference coreference cluster, to ensure a balanced data. By comparing the performance of the probing model using these two span representations, we can hypothesize to what extent that the proposed span representation in \textcite{joshi2019coref}  can capture coreference information. We will also experiment with mention span separation distance to see how the probing model performs and whether if there is a degradation of accuracy and F1 score of the probing model with distant spans.

\subsection{Data} 
We use OntoNotes (English) generating the positive and negative examples for the probing model. OntoNotes (English) is a widely used dataset on coreference resolution from the CoNLL-2012 shared task. It consists of about one million words of newswire, magazine articles, broadcast news, broadcast conversations, web data and conversational speech data, and the New Testament. The main evaluation is the average F1 score of three metrics – \textit{MUC}, $B^3$ and $CEAF_ \phi4$ on the test set according to the official CoNLL-2012 evaluation scripts.


\section{Experiments}

We evaluate the two models on the English OntoNotes 5.0 dataset \cite{conll}. 


\subsection{Implementation and Hyperparameters}. We extract the span representations from the coreference model by \textcite{joshi2019coref}. The model has been fine tuned on OntoNotes English data for 20 epochs using a dropout of 0.3, and learning rates of $1 * 10 ^{-5}$ and $2 * 10 ^{-4} $ with linear decay for the BERT parameters and the task parameters respectively. A batch size of 1 document has been used. 


\subsection{Baseline}




\section{Results}

\begin{itemize}
\item some nice plots here
\item some tables here
\end{itemize}

\section{Discussion}

\begin{itemize}
\item long long analysis of what we've seen
\item generalisations, parallels
\item what do this results tell us about coref and nlp in general
\item discussion of why they are the way the are
\end{itemize}

\subsection{Future Work}
\begin{itemize}
\item everything we don't have the time for
\item mention other corpora that could be used for finetuning (Winogrande, GAP...)

\end{itemize}

\section{Conclusion}

\begin{itemize}
\item so what have we learned about coref in general and local dependencies in particular
\end{itemize}

\printbibliography

\end{document}

% https://www.overleaf.com/learn/latex/Articles/Getting_started_with_BibLaTeX

%\textcite{joshi2019coref} bert for coref

%\textcite{lee2018higher} Higher-Order

%\textcite{lee2017end} End-to-End

%\textcite{devlin2019bert} Bert

